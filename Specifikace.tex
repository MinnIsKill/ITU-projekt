\documentclass[a4paper, 11pt, twocolumn]{article}

\usepackage[czech]{babel}
\usepackage[utf8]{inputenc}
\usepackage[IL2]{fontenc}
\usepackage[left=1.5cm, top=2.5cm, text={18cm, 25cm}]{geometry}

\usepackage{times}
\usepackage{amsthm}
\usepackage{amssymb}
\usepackage{amsmath}
\usepackage{stackrel}

\theoremstyle{definition}
\newtheorem{definition}{Definice}

\theoremstyle{plain}
\newtheorem{sentence}{Věta}



\begin{document}

	\begin{titlepage}
		\begin{center}
			{\Huge\textsc{Fakulta informačních technologií \\ \vspace*{0.2cm} Vysoké učení technické v~Brně}}\\
			\vspace{\stretch{0.382}} 
			\huge {Specifikace zadání a uživatelských požadavků} \\
			\Large {Tvorba uživatelských rozhraní} \\
			\vspace{\stretch{0.618}}
		\end{center}

		\Large{\hfill Vojtěch Kališ (xkalis03)} \\
		\Large{2021 \hfill Jan Lutonský (xluton02)}
	\end{titlepage}
	
	\twocolumn[{\centering{\Large \textbf{Individuální průzkum}\par}
	{\large Vojtěch Kališ, xkalis03@stud.fit.vutbr.cz\par}}
	\par\vspace*{0.8cm}]

	\section*{\large{Téma -- mobilní aplikace Time Planner}}
	\vspace*{-0.2cm}
	Jakožto téma jsem se rozhodl vybrat nějakou aplikaci, jejíž funkcionalita spočívá ve vytváření plánů, upomínek či jakéhokoliv jiného druhu poznámek za
	účelem zapsání si či připomenutí důležité události, schůze, nebo i~trivialit jako co je potřeba koupit při příští návštěvě obchodu apod.; čili jakýkoliv 
	například digitální kalendář, poznámkový blok a~jiné aplikace umožňující výše popsané. Zároveň proběhla snaha najít takovou aplikaci, na níž byly na
	první pohled patrné chyby z~pohledu GUI a~uživatelského procesu, a~to za účelem zjednodušení analýzy problémů dané aplikace a~následného návrhu
	jejich řešení.
	
	\section*{\large{Způsob zkoumání uživatele}}
	\vspace*{-0.2cm}
	Jako subjekt pro svůj uživatelský průzkum jsem si vybral svého bratra, jemuž jsem aplikaci na chvíli předal a~následně jej podrobil několika předem 
	přichystaným stručným otázkám, které měly za účel zjistit jeho pocity z aplikace, připomínky, požadavky a jiné.

	\section*{\large{Požadavky uživatele}} 
	\vspace*{-0.2cm}
	\begin{itemize}
		\item Větší přehlednost a jednoduchost použití
		\vspace{-0.2cm}
		\item Méně funkcí, zanechat pouze ty potřebné
		\vspace{-0.2cm}
		\item Možnost změny barevného scématu a pozadí
	\end{itemize}

	\section*{\large{Způsob používání současné aplikace}}
	
	\vspace*{-0.2cm}

	\section*{\large{Identifikované potíže}}
	\vspace*{-0.2cm}

	\section*{\large{Vlastní návrh nového řešení}}
	\vspace*{-0.2cm}


	\newpage


	\twocolumn[{\centering{\Large \textbf{Individuální průzkum}\par}
	{\large Jan Lutonský, xluton02@stud.fit.vutbr.cz\par}}
	\par\vspace*{0.8cm}]

	\section*{Téma -- školní informační systém}
	
	
	\section*{Způsob zkoumání uživatele}

	\section*{Požadavky uživatele}

	\section*{Způsob používání současné aplikace}

	\section*{Identifikované potíže}

	\section*{Vlastní návrh nového řešení}


	\newpage


	\twocolumn[{\centering{\huge \textbf{Specifikace zadání a uživatelských požadavků}\par}
	{\Large téma Školní informační systém\par}}
	\par\vspace*{0.8cm}]

	
	\section*{wow}	

\end{document}